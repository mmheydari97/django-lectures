\documentclass{beamer}

\mode<presentation> {
	
	\usetheme{Boadilla}
	\usecolortheme{seagull}  %plain
}
\usepackage{hyperref}
\usepackage{listings}

\usepackage{graphicx} % Allows including images
\usepackage{booktabs} % Allows the use of \toprule, \midrule and \bottomrule in tables

%----------------------------------------------------------------------------------------
%	TITLE PAGE
%----------------------------------------------------------------------------------------

\title[Django Workshop]{Django Workshop Session 1} % The short title appears at the bottom of every slide, the full title is only on the title page

\author{Advanced Programming} % Your name
\institute[AUT] % Your institution as it will appear on the bottom of every slide, may be shorthand to save space
{
	Amirkabir University of Technology\\ % Your institution for the title page
	\medskip
	\textit{} % Your email address
}
\date{Spring 2019} % Date, can be changed to a custom date

\begin{document}
	
	\begin{frame}
		\titlepage % Print the title page as the first slide
	\end{frame}

	\begin{frame}
	\frametitle{Overview} % Table of contents slide, comment this block out to remove it
	\tableofcontents % Throughout your presentation, if you choose to use \section{} and \subsection{} commands, these will automatically be printed on this slide as an overview of your presentation
	\end{frame}

%	PRESENTATION SLIDES

\section{Venv}
\subsection{What is Venv} 
\begin{frame}
\frametitle{What is Venv}
	virtual environment or simply venv allows you to have a virtual installation of python and packages on your computer.
\end{frame}

\subsection{Why Using Venv}
\begin{frame}
\frametitle{Why Using Venv}
\begin{itemize}
	\item Packages change and get updated often.
	\item Some of these changes break backwards compatibility.
	\item You may want to test out new features.
	\item However, you can't take down your website every time a package gets updated.
	\item So we will use venv for package (dependency) management.
\end{itemize}
\end{frame}

\subsection{How to Create a Venv}
\begin{frame}[fragile]
\frametitle{How to Create a Venv}
\lstset{language=Bash}         
\begin{lstlisting}[frame=single]
$ cd <project directory>
$ virtualenv -p <address of python> <name>
$ source ./<name>/bin/activate
$ pip install <packages>
$ pip freeze -l > requirements.txt **
$ ...
$ deactivate
\end{lstlisting}
it is also possible to use IDE tools related to virtual environments.
\\ ** We can also create requirements.txt file manually in order to avoid excess package installations.
\end{frame}

\section{Create Project}
\begin{frame}[fragile]
\frametitle{How to create django projects}
\lstset{language=Bash}         
\begin{lstlisting}[frame=single]
$ ... venv is activated with django installed
$ django-admin startproject <project name>
$ cd <project name>
$ django-admin startapp <app name>
$ ... magic happens
$ python manage.py migrate
$ python manage.py makemigrations
$ python manage.py migrate
$ python manage.py runserver
\end{lstlisting}
\end{frame}

\begin{frame}
	\frametitle{Some files generated}
	\begin{figure}
		\includegraphics[width=0.7\linewidth]{Pics/h69E992E0.jpeg}
	\end{figure}
\end{frame}


\section{Project}
\subsection{init}
\begin{frame}
	\frametitle{\_\_init\_\_.py}
	\begin{itemize}
		\item This is a blank python script that let's our directory to be treated as a python package.
	\end{itemize}
\end{frame}

\subsection{settings}
\begin{frame}
	\frametitle{settings.py}
	This is a where all of our project settings exists including:
	\begin{itemize}
		\item Base directory, templates and static files directories
		\item installed apps (defaults and defined apps)
		\item database engine and file
		\item password validation methods
		\item language and timezone used in project
	\end{itemize}
\end{frame}

\subsection{urls}
\begin{frame}
\frametitle{urls.py}
\begin{itemize}
	\item This is a Python scrip that will store all the url \underline{patterns} for our project.
	\item Basically the different pages of our application.
	\item linking methods implemented in business logic to urls generated by user interactions with project.
\end{itemize}
\end{frame}

\subsection{wsgi}
\begin{frame}
	\frametitle{wsgi.py}
	\begin{itemize}
		\item Wsgi stands for web server gateway interface.
		\item It helps us deploy our web app to product.
	\end{itemize}
\end{frame}

\subsection{manage}
\begin{frame}
	\frametitle{manage.py}
	\begin{itemize}
		\item Simply manages our project.
		\item We use it to run some of our commands as we build our project a lot.
	\end{itemize}
\end{frame}


\section{Application}
\begin{frame}
	\frametitle{App vs. Project}
	\begin{block}{App? Project?}
		A Python package provides a way of \underline{grouping} related Python code for easy reuse. A package contains one or more files of Python code (also known as “modules”).
		\newline
		A package can be imported with import foo.bar or from foo import bar. For a directory (like polls) to form a package, it must contain a special file \_\_init\_\_.py, even if this file is empty.
		\newline
		A Django application is just a Python package that is \underline{specifically intended} for use in a Django project. An application may use common Django conventions, such as having models, tests, urls, and views submodules.
		\newline
		Later on we use the term packaging to describe the process of making a Python package easy for others to install. It can be a little confusing, we know.
		
	\end{block}
\end{frame}

\subsection{admin}
\begin{frame}
	\frametitle{admin.py}
	\begin{itemize}
		\item We can register our models here
		\item Changing what to show and how to show them in admin interface.
	\end{itemize}
\end{frame}

\subsection{apps}
\begin{frame}
\frametitle{apps.py}
	
	\begin{itemize}
		\item Here we can place application specific configuration.
		\item We can also register our app by setting using apps.py.
	\end{itemize}
\end{frame}

\subsection{models}
\begin{frame}
	\frametitle{models.py}
	\begin{itemize}
		\item Here we can store our application data model.
		\item Classes stand for tables, attributes for fields etc.
		\item What entities are included in our application, how they are related and what attributes each of them has.
	\end{itemize}
\end{frame}

\subsection{test}
\begin{frame}
	\frametitle{test.py}
	\begin{itemize}
		\item We can design methods and classes to test our code automatically.
		\item Not included in our lectures.
	\end{itemize}
\end{frame}

\subsection{views}
\begin{frame}
	\frametitle{views.py}
	\begin{itemize}
		\item This is where we have functionalities of our application.
		\item handling requests and return responses
		\item Two methodologies: class based views (CBV) and function based views (FBV).
	\end{itemize}
\end{frame}

\subsection{templates}
\begin{frame}[fragile]
\frametitle{templates}
\begin{itemize}
	\item Templates are a key part to understanding how django really works and interacts with our website.
	\item We have some template tags to produce dynamic data.
	\item it goes under our top level directory that is:
	\begin{itemize}
		\item 
		\lstset{language=Bash}
		\begin{lstlisting}[frame=single]
<Project_dir>/templates/<App_dir>\end{lstlisting}
		
	\end{itemize}
\end{itemize}
\end{frame}

\subsection{Template tags}
\begin{frame}
	\frametitle{Template tags}
	There are some template tags by help of which we can insert dynamic content into HTML document. We will handle them in views.py . It might be surprising but we can define our own template tags.
	\begin{itemize}
		\item variables: \{\{ variable name \}\}
		\item extending: \{\% extends "base.html" \%\}
		\item loop: \{\% for obj in list \%\} body \{\% endfor \%\}
		\item condition: \{\% if condition \%\}\{\% elif condition \%\}\{\% else \%\}\{\% endif \%\}
		\item block: \{\% block name \%\} code \{\% endblock \%\}
		\item in forms: \{\% clrf\_token \%\}
		\item statics: \{\% load staticfiles \%\}
		\item source: src=\{\% static "a picture addr"\%\}
	\end{itemize}
\end{frame}

\subsection{Migrations}
\begin{frame}
	\frametitle{Migrations}
	\begin{itemize}
		\item This directory stores information about the database, related to the models and their changes.
		\item Usually ignored in commits.
	\end{itemize}
\end{frame}

\section{Notes}
\begin{frame}[fragile]
	\frametitle{Notes to Remember}
	\begin{enumerate}
		\item We usually add a urls.py to each application in order to make them more modular.
		\item Then the urls.py of project will include them and refer to them in order to resolve a url address.
		\lstset{language=python}
\begin{lstlisting}[frame=single]
from django.conf.urls import include
urlpatterns = [ ...
url(r'<a_pattern>/', include('<app_name>.urls')),
... ]
\end{lstlisting}
		\item We optionally make a forms.py script and separate our forms from other views in order to increase readability.
	\end{enumerate}
\end{frame}


\section{Approach}
\begin{frame}
	\frametitle{The Approach to build a project}
	It's up to the developer to take any approaches, however the steps below are recommended:
	\begin{enumerate}
		\item Designing our models carefully because it should change rarely and other parts are highly related on it.
		\item Implementing a functionality or action of our project in views.
		\item Mapping a url to the view and creating a template file.
		\item go to 2.
	\end{enumerate}
\end{frame}

\section{Static Files}

\begin{frame}
	\frametitle{Static Files}
	There are two ways to include static files like images in our project:
	\begin{itemize}
		\item fetching data from the database. (not covered)
		\item defining a folder and get images from there.
	\end{itemize}
	There will be folders named: css, js, images etc. in a directory named static, located in project directory.
	\\We should locate STATIC\_DIR at the end of settings.py.
\end{frame}

\begin{frame}
\Huge{\centerline{The End}}
\end{frame}

\end{document} 