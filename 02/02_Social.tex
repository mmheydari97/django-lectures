\documentclass{beamer}

\mode<presentation> {
	
	\usetheme{Boadilla}
	\usecolortheme{seagull}  %plain
}
\usepackage{hyperref}
\usepackage{listings}

\usepackage{graphicx} % Allows including images
\usepackage{booktabs} % Allows the use of \toprule, \midrule and \bottomrule in tables

%----------------------------------------------------------------------------------------
%	TITLE PAGE
%----------------------------------------------------------------------------------------

\title[Django Workshop]{Django Workshop Session 3} % The short title appears at the bottom of every slide, the full title is only on the title page

\author{Advanced Programming} % Your name
\institute[AUT] % Your institution as it will appear on the bottom of every slide, may be shorthand to save space
{
	Amirkabir University of Technology\\ % Your institution for the title page
	\medskip
	\textit{} % Your email address
}
\date{Spring 2019} % Date, can be changed to a custom date

\begin{document}
	
	\begin{frame}
		\titlepage % Print the title page as the first slide
	\end{frame}

	\begin{frame}
	\frametitle{Overview} % Table of contents slide, comment this block out to remove it
	\tableofcontents % Throughout your presentation, if you choose to use \section{} and \subsection{} commands, these will automatically be printed on this slide as an overview of your presentation
	\end{frame}

%	PRESENTATION SLIDES

\section{Project 2 intro}
\begin{frame}
\frametitle{Social project}
	\begin{itemize}
		\centering
		\large
		\item In this project we will create a social community site.
	\end{itemize}
\end{frame}

\subsection{pages}
\begin{frame}
\frametitle{Pages}
\begin{itemize}
	\item A sign up page.
	\item A Login page.
	\item A page for list of groups.
	\item A page for publishing new post.
	\item A home page.
\end{itemize}
\end{frame}

\subsection{features}
\begin{frame}
	\frametitle{features}
	\begin{itemize}
		\item \color{green}Multiple applications.
		\color{black}
		\item Multiple users and authorizations.
		\item Users can create groups.
		\item They can publish posts in groups.
		\item They can also join and leave other groups.
		\item Link user profiles with @ symbol.
	\end{itemize}
\end{frame}

\section{Part 1}
\begin{frame}
	\frametitle{Part 1}
	\begin{itemize}
		\item Create django project.
		\item Add apps and initial run.
		\item Set up essential files.
	\end{itemize}
\end{frame}

\section{Part 2}
\begin{frame}
	\frametitle{Part 2}
	\begin{itemize}
		\item Set up models.
		\item import libraries.
		
	\end{itemize}
\end{frame}

\section{Part 3}
\begin{frame}
	\frametitle{Part 3}
	\begin{itemize}
		\item Set up forms.
		\item Style widgets using Meta.
		
	\end{itemize}
\end{frame}

\section{Part 4}
\begin{frame}[fragile]
	\frametitle{Part 4}
	\begin{itemize}
		\item Create views and templates.
		\item Connect them to urls.
		\item Start using Class Based Views.
		\item \href{https://docs.djangoproject.com/en/2.1/ref/models/querysets/#id4}{Field Lookups doc.}
	\end{itemize}
	
\end{frame}

\section{Part 5}
\begin{frame}
	\frametitle{Part 5}
	\begin{itemize}
		\item Complete templates.
		\item Create Function Based Views.
		
	\end{itemize}
\end{frame}


\section{Part 6}
\begin{frame}
	\frametitle{Part 6}
	\begin{itemize}
		\item Create simple authentication system.
		
	\end{itemize}
\end{frame}

\section{Part 7}
\begin{frame}[fragile]
	\frametitle{Part 7}
	\begin{itemize}
		\item Review templates.
		\item
		\href{https://getbootstrap.com/docs/3.3/getting-started/}{Bootstrap 3}
		
		\item
		\href{https://github.com/yabwe/medium-editor}{Medium-editor}
		
		\item
		\href{https://codepen.io/thapliyalshivam/pen/dvgXVO}{Colorful css}
		\item Google fonts
	\end{itemize}
\end{frame}


\section{Ajax}
\begin{frame}[fragile]
	\frametitle{Ajax}
	\begin{itemize}
		\item A for asynchronous.
		\item JSON
		\item jQuery
		\item
		\href{https://simpleisbetterthancomplex.com/tutorial/2016/08/29/how-to-work-with-ajax-request-with-django.html}{Example}
		
	\end{itemize}
\end{frame}

%\begin{frame}[fragile]
%\frametitle{How to Create a Venv}
%\lstset{language=Bash}         
%\begin{lstlisting}[frame=single]
%$ cd <project directory>
%$ virtualenv -p <address of python> <name>
%$ source ./<name>/bin/activate
%$ pip install <packages>
%$ pip freeze -l > requirements.txt **
%$ ...
%$ deactivate
%\end{lstlisting}
%it is also possible to use IDE tools related to virtual environments.
%\\ ** We can also create requirements.txt file manually in order to avoid excess package installations.
%\end{frame}


\begin{frame}
\Huge \centering The End
\end{frame}

\end{document} 