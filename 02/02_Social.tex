\documentclass{beamer}

\mode<presentation> {
	
	\usetheme{Boadilla}
	\usecolortheme{seagull}  %plain
}
\usepackage{hyperref}
\usepackage{listings}

\usepackage{graphicx} % Allows including images
\usepackage{booktabs} % Allows the use of \toprule, \midrule and \bottomrule in tables

%----------------------------------------------------------------------------------------
%	TITLE PAGE
%----------------------------------------------------------------------------------------

\title[Django Workshop]{Django Workshop Session 3} % The short title appears at the bottom of every slide, the full title is only on the title page

\author{Advanced Programming} % Your name
\institute[AUT] % Your institution as it will appear on the bottom of every slide, may be shorthand to save space
{
	Amirkabir University of Technology\\ % Your institution for the title page
	\medskip
	\textit{} % Your email address
}
\date{Spring 2019} % Date, can be changed to a custom date

\begin{document}
	
	\begin{frame}
		\titlepage % Print the title page as the first slide
	\end{frame}

	\begin{frame}
	\frametitle{Overview} % Table of contents slide, comment this block out to remove it
	\tableofcontents % Throughout your presentation, if you choose to use \section{} and \subsection{} commands, these will automatically be printed on this slide as an overview of your presentation
	\end{frame}

%	PRESENTATION SLIDES

\section{Project 2 intro}
\begin{frame}
\frametitle{Social project}
	\begin{itemize}
		\centering
		\large
		\item In this project we will create a social community site.
		\item Focus on templates
		\item Debugging
	\end{itemize}
\end{frame}

\subsection{pages}
\begin{frame}
\frametitle{Pages}
\begin{itemize}
	\item A sign up page.
	\item A Login page.
	\item A page for list of groups.
	\item A page for publishing new post.
	\item A home page.
\end{itemize}
\end{frame}

\subsection{features}
\begin{frame}
	\frametitle{features}
	\begin{itemize}
		\item \color{green}Multiple applications.
		\color{black}
		\item Multiple users and authorizations.
		\item Users can create groups.
		\item They can publish posts in groups.
		\item They can also join and leave other groups.
		\item Link user profiles with @ symbol.
	\end{itemize}
\end{frame}

\section{Part 1}
\begin{frame}
	\frametitle{Part 1}
	\begin{itemize}
		\item Create django project.
		\item Add apps and initial run.
		\item Set up essential files.
	\end{itemize}
\end{frame}

\section{Part 2}
\begin{frame}
	\frametitle{Part 2}
	\begin{itemize}
		\item Set up accounts app.
		\item Create base and index templates.
		\item Fill in models.py for users.
		\item Create forms.py.
		\item Handle sign up.
	\end{itemize}
\end{frame}

\section{Part 3}
\begin{frame}
	\frametitle{Part 3}
	\begin{itemize}
		\item Focus on templates in accounts.
		\item Install bootstrap3 for django.
		
		
	\end{itemize}
\end{frame}

\section{Part 4}
\begin{frame}
	\frametitle{Part 4}
	\begin{itemize}
		\item Create redirect pages for login and logout.
	\end{itemize}
	
\end{frame}

\section{Part 5}
\begin{frame}
	\frametitle{Part 5}
	\begin{itemize}
		\item Create posts and groups simultaneously.
		\item Set up related files.
		
	\end{itemize}
\end{frame}

\section{Part 6}
\begin{frame}
\frametitle{Part 6}
\begin{itemize}
	\item Create models.py for posts and groups.
	\item Many to many relation.
	\item Install django-misaka.
	
\end{itemize}
\end{frame}

\section{Part 7}
\begin{frame}
\frametitle{Part 7}
\begin{itemize}
	\item Create views and templates for groups.
	\item Use include template tag.
\end{itemize}
\end{frame}

\section{Part 8}
\begin{frame}
\frametitle{Part 8}
\begin{itemize}
	\item Handle urls of groups.
	\item Register groups in admin.
	
\end{itemize}
\end{frame}

\section{Part 9}
\begin{frame}[fragile]
\frametitle{Part 9}
\begin{itemize}
	\item Create views and templates for posts.
	\item Install django-braces.
	\item Register posts in admin.
	\item \href{https://docs.djangoproject.com/en/2.1/ref/models/querysets/}{prefetch\_related}
	
\end{itemize}
\end{frame}


\section{Part 10}
\begin{frame}
\frametitle{Part 10}
\begin{itemize}
	\item Create views and templates for posts.
	\item Install django-humanize.
\end{itemize}
\end{frame}


%\begin{frame}[fragile]
%\frametitle{How to Create a Venv}
%\lstset{language=Bash}         
%\begin{lstlisting}[frame=single]
%$ cd <project directory>
%$ virtualenv -p <address of python> <name>
%$ source ./<name>/bin/activate
%$ pip install <packages>
%$ pip freeze -l > requirements.txt **
%$ ...
%$ deactivate
%\end{lstlisting}
%it is also possible to use IDE tools related to virtual environments.
%\\ ** We can also create requirements.txt file manually in order to avoid excess package installations.
%\end{frame}


\begin{frame}
\Huge \centering Any Questions?
\end{frame}

\begin{frame}
\Huge \centering The End
\end{frame}


\end{document} 